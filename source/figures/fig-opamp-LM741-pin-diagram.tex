\begin{circuitikz}
    \draw (0,0) node[dipchip, num pins=8, hide numbers, external pins width=0, circuitikz/chips/scale=1](C){};
    \foreach \pin in {1,...,4} \node[font=\tiny, left] at(C.bpin \pin) {\pin};
    \foreach \pin in {5,...,8} \node[font=\tiny, right] at(C.bpin \pin) {\pin};
    \node [left, xshift=-2mm, font=\tiny, align=center] at (C.pin 1) {Offset\\Null};
    \node [left, xshift=-2mm, font=\tiny, align=center] at (C.pin 2) {$V_-$};
    \node [left, xshift=-2mm, font=\tiny, align=center] at (C.pin 3) {$V_+$};
    \node [left, xshift=-2mm, font=\tiny, align=center] at (C.pin 4) {$V_{s-}$};
    \node [right, xshift=+2mm, font=\tiny, align=center] at (C.pin 5) {Offset\\Null};
    \node [right, xshift=+2mm, font=\tiny, align=center] at (C.pin 6) {$V_\text{out}$};
    \node [right, xshift=+2mm, font=\tiny, align=center] at (C.pin 7) {$V_{s+}$};
    \node [right, xshift=+2mm, font=\tiny, align=center] at (C.pin 8) {Not\\Used};
    % Place op-amp schematic connections inside chip pin diagram
    \path ($(C.pin 6)!0.5!(C.pin 7)$) coordinate (midway-6-7)
            (midway-6-7 -| C.north) node[op amp, scale=0.4](opamp){};
    % connect pins
    \draw (C.pin 2) -| (opamp.-);
    \draw (C.pin 3) -| (opamp.+);
    \draw (C.pin 6) -| (opamp.out);
    \draw (C.pin 4) -| (opamp.down);
    \draw (C.pin 7) -| (opamp.up);
\end{circuitikz}